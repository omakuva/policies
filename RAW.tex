%!TEX TS-program = xelatex
%!TEX encoding = UTF-8 Unicode

\documentclass[11pt,oneside,titlepage,final,a4paper]{article}

%\usepackage{geometry} 
%\geometry{a4paper}

%\usepackage[parfill]{parskip}

\usepackage{polyglossia} %
\setdefaultlanguage%[⟨options⟩]
{american} %
\setotherlanguages{german,italian,french,russian}

%\usepackage[american,german,%frenchb,
%italian]{babel}

\usepackage{ucs}
% \usepackage[utf8x]{inputenc} %
% \usepackage[T1]{fontenc}
\usepackage{lmodern}
\usepackage{textcomp}
\usepackage{fontspec} 
\usepackage{xunicode,xltxtra,url,%parskip,
mdwlist}
\usepackage{wasysym}
\usepackage[normalem]{ulem}

%\usepackage{expl3}

% \usepackage{footmisc}
% \renewcommand{\thefootnote}{\fnsymbol{footnote}}

\title{%
  % THE GNOSTIC SOCIETY LIBRARY \\
  % The Nag Hammadi Library: \\
  % The Gospel of Thomas \\
  % Translated by Thomas O. Lambdin
}

\author{Emanuele Rodo}

\defaultfontfeatures{Mapping=tex-text} 
\setromanfont[Mapping=tex-text]{%TeX %
  %Junicode%
  %Latin Modern Roman%
  Hoefler Text%
  % TeX Gyre Schola%
  % Big Caslon%
  % Didot%
  %Latin Modern Mono Prop% 
} 
\setsansfont[Scale=MatchLowercase,Mapping=tex-text]{Gill Sans} 
\setmonofont[Scale=MatchLowercase]{Andale Mono}

\usepackage[bookmarks,colorlinks=true,linktocpage=true,citecolor=blue,%
unicode=true,%
pdfauthor={Emanuele Rodo},pdftitle={A}]{hyperref}

\begin{document}

{\sc A.}

\smallskip

Tentatively, we orient a possible discourse centered on
\textit{media}, -- a \textit{constitutive} network of these, -- their
digitalized, authority preserving governance, record, and
delivery;\footnote{States beyond space-time confinments,
  i.e.~implementing current technological advances, such as
  noise-induced phase transitions, area-laws and holography in a
  quantum field theoretical setting.%, from within an algebraic
  % viewpoint.
} -- its possibility, in fact -- from a free software, ISO, and
creative commons perspective.

\textit{Assessment} is a peculiar element in nowadays informational
propaganda.

Aim is to discuss current issues such as that of \textit{identity} in
storage devices; unit transmissions, law and currency of digital
significance, where issues are about when the border between knowledge
and a-knowledgement merge.\footnote{Conforming to this picture, legal
  questions then arise about the nature of \textit{this} border -- in
  fact, this being the only element capable of grant for the
  legitimacy of one and the other part.} Thus, focus of this
presentation is:
\begin{quote} % 
  {\sc Can a signal -- of whatever nature -- 
  %together 
    with the noise it implies, be given the %right
    status of information only if it is
    re-cognized?}\footnote{References to signal and noise are targeted
    to within Landauer's analysis: -- perspectives then unfold if
    subjected to current advances in area-law and full counting
    statistics studies.}%
\end{quote}%
While the affinities with the concept of \textit{trace} in Derrida's
work are relevant,\footnote{Over and over, main reference is the
  \textit{Grammatology}, for a non-historical non-semantic permitted
  allowance of the key principle of \textit{letter}, here <<grapheme>>
  -- i.e., character and sign.}  emphasis is placed here onto where
the \textit{local} legitimacy boundary establishes -- e.g., as to: the
author of the noise: the emitter? the governance, if a signal
satisfies some proper standards? to a later (distant) authority which
may credit signatures their right?\footnote{It is for us interesting,
  beside all the difficulties mentioned, to confront a period --
  \textit{this} -- in which the <<text's consciousness>>, the
  articulated genealogical representation of it, can be fully
  exploited, without special reference to personal credits. -- Since:
  <<What is a lineage in the order of discourse and text? If in a
  rather conventional way I call by the name of \textit{discourse} the
  present, living, conscious \emph{representation} of a \emph{text}
  within the experience of the person who writes or reads it, and if
  the text constantly goes beyond this representation by the entire
  system of its resources and its own laws, then the question of
  genealogy exceeds by far the possibilities that are at present given
  for its elaboration.>> (p.~101). Social coding possibilities are
  current answers to these questions.}

Then, is information to be considered \textit{alive} if it is free
propagate; -- if yes, what is then its medium?

Is legitimacy only assessed by the immanent, all encompassing Spirit,
{\sc KEY} for any future, current and past understanding of its
manifestation?

\end{document}
